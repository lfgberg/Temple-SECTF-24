\begin{fullwidth}
\section{NIST Phish Scale}
\label{sec:NIST Phish Scale}

The NIST Phish scale was developed by NIST to identify phishing attacks and evaluate the effectiveness of training to combat these attacks. The scale evaluates the complexity and quality of a phishing attack so that organizations can better train their employees and/or clients to identify these attacks \autocite{Dawkins:2023}.

NIST considers the following criteria in an email when assessing a phishing attack:
\begin{enumerate}
    \item \textbf{Error}: Spelling errors, grammatical errors, and inconsistencies.
    \item \textbf{Technical Indicator}: Attachment types, sender email, sender information, hyperlinks, and domains used.
    \item \textbf{Visual Presentation Indicator}: Professionalism of the email, company logos, and other visual elements that would be expected in a cooperate email. 
    \item \textbf{Language and Content}: Threats presented by the email writer, urgency, lack of details, and/or irrelevant details.
    \item \textbf{Common Tactic}: Such as too good to be true offers, special treatment, and/or posing as a friend/colleague/employer.
\end{enumerate}

NIST also considers the premise alignment of the communications to indicate how difficult it is for a victim to detect. Premise alignment is a measure of how closely an email matches the work roles or responsibilities of an email’s recipient or organization. The stronger an email’s premise alignment, the more difficult it is to detect as a phish. Inversely, the weaker an email’s premise alignment, the easier it is to detect as a phish \autocite{Dawkins:2023}.

The following premise alignment attributes are assessed on the NIST phish scale:
\begin{enumerate}
    \item \textbf{Mimics a Workplace Process or Practice}: The closer a phishing email mimics how an organization acts, the more likely the target is to fall for the phishing attack.
    \item \textbf{Has Workplace Relevance}: If the attack takes the target's job position and workplace access into account, they target is more likely to find the email legitimate.
    \item \textbf{Aligns With Situations or Events}: Should the message be timely to events occurring in the target's life or organization, the target is more likely to believe that the message is legitimate.
    \item \textbf{Engenders Concern Over Consequences}: If the email is threatening the target to take action or face consequences, the target is more likely to perform the requested actions in order to avoid the consequences, even if the consequences do not exist.
    \item \textbf{Training/Experience of Recipient With Phishing Emails}: When the target is trained in identifying phishing attacks, or has previously been a victim to phishing attacks, they are more likely to identify the attack and stop it.
\end{enumerate}

\end{fullwidth}