\begin{fullwidth}
\section{MITRE ATT\&CK Framework}
\label{sec:MITRE Attack Framework}
The MITRE ATT\&CK Framework catalogs cybercriminal's tactics, techniques and procedures in each phase of their attack. This allows defenders to identify such attack methods and ensure that their defenses are capable of stopping such attacks \autocite{IBM}.

The attack framework is ordered chronologically, with 1 being the first phase of the attack and 14 being the last phase of the attack. Each phase of the MITRE ATT\&CK Framework is as follows:

\begin{enumerate}
    \item \textbf{Reconnaissance}: gather information about the target to plan for an attack.
    \item \textbf{Resource Development}: build and acquire resources to carry out the attack. This can include domains, web sites, and email servers.
    \item \textbf{Initial Access}: Exploit the target to get initial access to their environment.
    \item \textbf{Execution}: Run malware or malicious code on the exploited system.
    \item \textbf{Persistence}: Setup access to the system that will withstand reboots or system reconfiguration.
    \item \textbf{Privilege Escalation}: Gain access to accounts with higher privileges, such as an administrator.
    \item \textbf{Defense Evasion}: Avoid detection, such as anti-virus or intrusion detection systems.
    \item \textbf{Credential Access}: Gather usernames, passwords, and other credentials to expand access.
    \item \textbf{Discovery}: Explore and research the target's system to find systems that can be accessed or controlled to support an attack.
    \item \textbf{Lateral Movement}: Gain access to other resources in the environment.
    \item \textbf{Collection}: Gather target's data in the environment related to the goal.
    \item \textbf{Command and Control}: Establish covert/undetectable communications from the target's systems to the attacker that enable control over the target's system.
    \item \textbf{Exfiltration}: Steal data from the target's system.
    \item \textbf{Impact}: Interrupt the business' function and data.
\end{enumerate}
\end{fullwidth}