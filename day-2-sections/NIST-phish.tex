\begin{fullwidth}
\section{NIST Phish Scale} % Top level section

The \hyperref[sec:NIST Phish Scale]{NIST Phish Scale} is a tool developed by NIST to assist with the identification of Phishing scams \autocite{Dawkins:2023}. The team has applied the \hyperref[sec:NIST Phish Scale]{NIST Phish Scale} worksheet to Sam's initial employment email to help determine its legitimacy.

\subsection{Interview Offer Email}

\begin{figure*}[H] % Use the figure* environment for full width figures
    \centering
    \includegraphics[scale=.5]{assets/full-interview-email-1.png}
    \includegraphics[scale=.5]{assets/full-interview-email-2.png}
    \captionsetup{justification=centering}
    \caption{Sam's invitation to interview with "HealthComp"}
\end{figure*}

\subsubsection{Email Cues} % Second level section

\textbf{Part 1: Answer “yes” or “no” to the following questions}\\\medskip
\textit{Technical Indicators}
    \begin{description}
        \item[Q:] Is the sender’s name unrelated to the sender’s email address, including “reply-to” address?
        \item[A:] \textbf{Yes} - The email is sent from Kellie McDaniel, who is not referenced at all throughout the interview and hiring process.
        \item[Q:] Is a domain name used in the sender's email address plausibly similar to a
recognizable entity's domain?
        \item[A:] \textbf{Yes} - Although we don't have Kellie McDaniel's email address, the email later references davidbbondeson@healthcomp.live, which is similar to the real healthcomp.com domain.
    \end{description}

\textit{Visual Presentation Indicators}
\begin{description}
    \item[Q:] Are appropriate branding elements (text or logos) missing?
    \item[A:] \textbf{Yes} - The email does not contain any logos or branding other than the use of the HealthComp name.
    \item[Q:] Do the design and formatting of the email appear unprofessional?
    \item[A:] \textbf{Yes} - The email doesn't utilize professional formatting, it has a number of grammatical errors and is not in line with what's expected from a typical employer.
\end{description}

\textit{Language and Content}
\begin{description}
    \item[Q:] Is the email missing a generic greeting, such as a formal or informal
salutation?
    \item[A:] \textbf{No} - The email begins with "Dear Candidate", and ends with "Best regards"
    \item[Q:] Is the email missing personalization?
    \item[A:] \textbf{Yes} -  The email is addressed to "Candidate", and doesn't refer to Sam by name at all.
    \item[Q:] Is the message missing detail about the sender, such as sender or contact
information?
    \item[A:] \textbf{Yes} - The email does not mention the sender at all, it's signed from "HealthComp" and doesn't reference Kellie McDaniel.
\end{description}

\textit{Common Tactics}
\begin{description}
    \item[Q:] Does the message appear to be a work or business-related process?
    \item[A:] \textbf{Yes} - The message appears to be related to an interview offer at HealthComp.
    \item[Q:] Does the message appear to be from a friend, colleague, boss, other authority
entity, or other reputable authority entity?
    \item[A:] \textbf{No} - No real information is provided about the sender of the email, Kellie McDaniel, and it's generically signed "HealthComp".
\end{description}

Total number of "yes" responses: \textbf{7}

\textbf{Part 2: Tally the total number of times the following appear in the email}

\textit{Errors}
\begin{itemize}
    \item How many spelling errors are in the email? \textbf{1}
    \item How many grammar errors are in the email, including mismatched plurality? \textbf{16}
    \item How many inconsistencies are in the email? \textbf{3}
\end{itemize}

\textit{Technical Indicators}
\begin{itemize}
    \item How many potentially dangerous attachments are included? \textbf{0}
    \item How many times does text hide the true URL in a hyperlink? \textbf{0}
    \item How many links have a domain name plausibly similar to a to a recognizable
entity's domain? \textbf{1}
\end{itemize}

\textit{Visual Presentation Indicators}
\begin{itemize}
    \item How many branding elements (text or logos) appear to be an imitation? \textbf{0}
    \item How many branding elements (text or logos) appear to be out-of-date? \textbf{0}
    \item How many inappropriate security indicators or security icons are in the
email? \textbf{2}
\end{itemize}

\textit{Language and Content}
\begin{itemize}
    \item How many times is legal language used in the message, such as copyright
information, disclaimers, or tax information? \textbf{0}
    \item How many detailed aspects that are not central to the content are in the
message? \textbf{2}
    \item How many requests for sensitive information are in the email, including
personally identifying information or credentials? \textbf{0}
    \item How many times does the email express time pressure, including implied? \textbf{0}
    \item How many threats are included in the message, including implied threats? \textbf{0}
\end{itemize}

\textit{Common Tactics}
\begin{itemize}
    \item How many appeals does the email make to help others? \textbf{0}
    \item How many times does the email offer something that is too good to be true,
such as having won a contest, lottery, free vacation and so on? \textbf{1}
    \item Does the email offer anything personalized and unexpected just for you? \textbf{Yes (1)}
    \item How many times does the email offer something for a limited time? \textbf{0}
\end{itemize}

Sum of tallied cues: \textbf{27}\\\medskip
Total cue count from Part 1 and Part 2: \textbf{34}

\begin{table*}[H]
\centering
\begin{tabular}{|l|c|}
\hline
\rowcolor[HTML]{96BEE6} 
\multicolumn{1}{|c|}{\cellcolor[HTML]{96BEE6}\textbf{Total Cue Count}} & \textbf{Cue Category} \\ \hline
1 – 8 cues                                                             & Few (more difficult)  \\ \hline
9 – 14 cues                                                            & Some                  \\ \hline
15 or more cues                                                        & Many (less difficult) \\ \hline
\end{tabular}
\captionsetup{justification=centering}
\caption{Cue Category Mapping}
\end{table*}

Cue Category: \textbf{Many (less difficult)}

\subsubsection{Premise Alignment} % Second level section

\begin{itemize}
    \item How applicable is the email to workplace processes or practices for the target
audience? \textbf{6}
    \item How pertinent is the email’s premise to the roles and responsibilities of the
target audience? \textbf{0}
    \item How well does the email align to other situations or events, even those external
to the workplace? \textbf{6}
    \item How applicable is the email to concerns over potentially harmful ramifications
for not clicking the links or attachments? \textbf{0}
    \item How applicable is the email’s reflection of targeted training effects that would
lead to premise detection? Care must be taken to appropriately incorporate the
training or warning specificity, as transfer of learning is quite difficult. \textbf{8}
\end{itemize}

\begin{table*}[H]
\centering
\begin{tabular}{|l|c|}
\hline
\rowcolor[HTML]{96BEE6} 
\multicolumn{1}{|c|}{\cellcolor[HTML]{96BEE6}\textbf{Applicability Scale}} & \textbf{Applicability Score} \\ \hline
Extreme applicability, alignment, or relevancy                    & 8                   \\ \hline
Significant applicability, alignment, or relevancy                & 6                   \\ \hline
Moderate applicability, alignment, or relevancy                   & 4                   \\ \hline
Low applicability, alignment, or relevancy                        & 2                   \\ \hline
Not applicable, no alignment, or no relevancy                     & 0                   \\ \hline
\end{tabular}
\captionsetup{justification=centering}
\caption{Applicability Scale}
\end{table*}

Premise Alignment Rating: \textbf{4}

\begin{table*}[H]
\centering
\begin{tabular}{|l|c|}
\hline
\rowcolor[HTML]{96BEE6} 
\multicolumn{1}{|c|}{\cellcolor[HTML]{96BEE6}\textbf{Premise Alignment Rating}} & \textbf{Premise Alignment Category} \\ \hline
10 and below                                                                    & Weak                                \\ \hline
11 – 17                                                                         & Medium                              \\ \hline
18 and higher                                                                   & Strong                              \\ \hline
\end{tabular}
\captionsetup{justification=centering}
\caption{Premise Alignment Category Mapping}
\end{table*}

Premise Alignment Category: \textbf{Weak}

\subsubsection{Detection Difficulty} % Second level section

\begin{table*}[H]
\centering
\begin{tabular}{|l|l|l|}
\hline
\rowcolor[HTML]{96BEE6} 
\multicolumn{1}{|c|}{\cellcolor[HTML]{96BEE6}\textbf{Cues Category}} & \multicolumn{1}{c|}{\cellcolor[HTML]{96BEE6}\textbf{Premise Alignment Category}} & \textbf{Detection Difficulty} \\ \hline
                                                                     & Strong                                                                           & Very difficult                \\ \cline{2-3} 
                                                                     & Medium                                                                           & Very difficult                \\ \cline{2-3} 
\multirow{-3}{*}{Few (more difficult)}                               & Weak                                                                             & Moderately difficult          \\ \hline
                                                                     & Strong                                                                           & Very difficult                \\ \cline{2-3} 
                                                                     & Medium                                                                           & Moderately difficult          \\ \cline{2-3} 
\multirow{-3}{*}{Some}                                               & Weak                                                                             & Moderately to Least difficult \\ \hline
                                                                     & Strong                                                                           & Moderately difficult          \\ \cline{2-3} 
                                                                     & Medium                                                                           & Moderately difficult          \\ \cline{2-3} 
\multirow{-3}{*}{Many (less difficult)}                              & Weak                                                                             & Least difficult               \\ \hline
\end{tabular}
\captionsetup{justification=centering}
\caption{Detection Difficulty Mapping}
\end{table*}

Overall Detection Difficulty Rating: \textbf{Least difficult}

\subsection{Job Offer Email}

\begin{figure*}[H] % Use the figure* environment for full width figures
    \centering
    \includegraphics{assets/OfferEmail.png}
    \captionsetup{justification=centering}
    \caption{Sam's job offer email}
\end{figure*}

\subsubsection{Email Cues} % Second level section

\textbf{Part 1: Answer “yes” or “no” to the following questions}\\\medskip
\textit{Technical Indicators}
    \begin{description}
        \item[Q:] Is the sender’s name unrelated to the sender’s email address, including “reply-to” address?
        \item[A:] \textbf{Yes} - The email is sent from David Bondeson, although this was the initial point of contact for the interview, the individual conducting the text based interview was Gavin Manley.
        \item[Q:] Is a domain name used in the sender's email address plausibly similar to a
recognizable entity's domain?
        \item[A:] \textbf{Yes} - davidbbondeson@healthcomp.live, is similar to the real healthcomp.com domain.
    \end{description}

\textit{Visual Presentation Indicators}
\begin{description}
    \item[Q:] Are appropriate branding elements (text or logos) missing?
    \item[A:] \textbf{Yes} - The email does not contain any logos or branding other than the use of the HealthComp name.
    \item[Q:] Do the design and formatting of the email appear unprofessional?
    \item[A:] \textbf{Yes} - The email doesn't utilize professional formatting, it has a number of grammatical errors and is not in line with what's expected from a typical employer.
\end{description}

\textit{Language and Content}
\begin{description}
    \item[Q:] Is the email missing a generic greeting, such as a formal or informal
salutation?
    \item[A:] \textbf{No} - The email begins with "Hello Sam Gamgee", and ends with "Best regards"
    \item[Q:] Is the email missing personalization?
    \item[A:] \textbf{No} -  The email is addressed to Sam specifically.
    \item[Q:] Is the message missing detail about the sender, such as sender or contact
information?
    \item[A:] \textbf{Yes} - No contact information or sender name is indicated in the email.
\end{description}

\textit{Common Tactics}
\begin{description}
    \item[Q:] Does the message appear to be a work or business-related process?
    \item[A:] \textbf{Yes} - The message appears to be related to a job offer at HealthComp.
    \item[Q:] Does the message appear to be from a friend, colleague, boss, other authority
entity, or other reputable authority entity?
    \item[A:] \textbf{Yes} - The message appears to be from the "Employee Success Director" at HealthComp.
\end{description}

Total number of "yes" responses: \textbf{7}

\textbf{Part 2: Tally the total number of times the following appear in the email}

\textit{Errors}
\begin{itemize}
    \item How many spelling errors are in the email? \textbf{0}
    \item How many grammar errors are in the email, including mismatched plurality? \textbf{9}
    \item How many inconsistencies are in the email? \textbf{3}
\end{itemize}

\textit{Technical Indicators}
\begin{itemize}
    \item How many potentially dangerous attachments are included? \textbf{2}
    \item How many times does text hide the true URL in a hyperlink? \textbf{0}
    \item How many links have a domain name plausibly similar to a to a recognizable
entity's domain? \textbf{0}
\end{itemize}

\textit{Visual Presentation Indicators}
\begin{itemize}
    \item How many branding elements (text or logos) appear to be an imitation? \textbf{0}
    \item How many branding elements (text or logos) appear to be out-of-date? \textbf{0}
    \item How many inappropriate security indicators or security icons are in the
email? \textbf{0}
\end{itemize}

\textit{Language and Content}
\begin{itemize}
    \item How many times is legal language used in the message, such as copyright
information, disclaimers, or tax information? \textbf{0}
    \item How many detailed aspects that are not central to the content are in the
message? \textbf{0}
    \item How many requests for sensitive information are in the email, including
personally identifying information or credentials? \textbf{2}
    \item How many times does the email express time pressure, including implied? \textbf{0}
    \item How many threats are included in the message, including implied threats? \textbf{0}
\end{itemize}

\textit{Common Tactics}
\begin{itemize}
    \item How many appeals does the email make to help others? \textbf{0}
    \item How many times does the email offer something that is too good to be true,
such as having won a contest, lottery, free vacation and so on? \textbf{1}
    \item Does the email offer anything personalized and unexpected just for you? \textbf{Yes (1)}
    \item How many times does the email offer something for a limited time? \textbf{0}
\end{itemize}

Sum of tallied cues: \textbf{18}\\\medskip
Total cue count from Part 1 and Part 2: \textbf{25}

\begin{table*}[H]
\centering
\begin{tabular}{|l|c|}
\hline
\rowcolor[HTML]{96BEE6} 
\multicolumn{1}{|c|}{\cellcolor[HTML]{96BEE6}\textbf{Total Cue Count}} & \textbf{Cue Category} \\ \hline
1 – 8 cues                                                             & Few (more difficult)  \\ \hline
9 – 14 cues                                                            & Some                  \\ \hline
15 or more cues                                                        & Many (less difficult) \\ \hline
\end{tabular}
\captionsetup{justification=centering}
\caption{Cue Category Mapping}
\end{table*}

Cue Category: \textbf{Many (less difficult)}

\subsubsection{Premise Alignment} % Second level section

\begin{itemize}
    \item How applicable is the email to workplace processes or practices for the target
audience? \textbf{8}
    \item How pertinent is the email’s premise to the roles and responsibilities of the
target audience? \textbf{8}
    \item How well does the email align to other situations or events, even those external
to the workplace? \textbf{6}
    \item How applicable is the email to concerns over potentially harmful ramifications
for not clicking the links or attachments? \textbf{4}
    \item How applicable is the email’s reflection of targeted training effects that would
lead to premise detection? Care must be taken to appropriately incorporate the
training or warning specificity, as transfer of learning is quite difficult. \textbf{6}
\end{itemize}

\begin{table*}[H]
\centering
\begin{tabular}{|l|c|}
\hline
\rowcolor[HTML]{96BEE6} 
\multicolumn{1}{|c|}{\cellcolor[HTML]{96BEE6}\textbf{Applicability Scale}} & \textbf{Applicability Score} \\ \hline
Extreme applicability, alignment, or relevancy                    & 8                   \\ \hline
Significant applicability, alignment, or relevancy                & 6                   \\ \hline
Moderate applicability, alignment, or relevancy                   & 4                   \\ \hline
Low applicability, alignment, or relevancy                        & 2                   \\ \hline
Not applicable, no alignment, or no relevancy                     & 0                   \\ \hline
\end{tabular}
\captionsetup{justification=centering}
\caption{Applicability Scale}
\end{table*}

Premise Alignment Rating: \textbf{20}

\begin{table*}[H]
\centering
\begin{tabular}{|l|c|}
\hline
\rowcolor[HTML]{96BEE6} 
\multicolumn{1}{|c|}{\cellcolor[HTML]{96BEE6}\textbf{Premise Alignment Rating}} & \textbf{Premise Alignment Category} \\ \hline
10 and below                                                                    & Weak                                \\ \hline
11 – 17                                                                         & Medium                              \\ \hline
18 and higher                                                                   & Strong                              \\ \hline
\end{tabular}
\captionsetup{justification=centering}
\caption{Premise Alignment Category Mapping}
\end{table*}

Premise Alignment Category: \textbf{Strong}

\subsubsection{Detection Difficulty} % Second level section

\begin{table*}[H]
\centering
\begin{tabular}{|l|l|l|}
\hline
\rowcolor[HTML]{96BEE6} 
\multicolumn{1}{|c|}{\cellcolor[HTML]{96BEE6}\textbf{Cues Category}} & \multicolumn{1}{c|}{\cellcolor[HTML]{96BEE6}\textbf{Premise Alignment Category}} & \textbf{Detection Difficulty} \\ \hline
                                                                     & Strong                                                                           & Very difficult                \\ \cline{2-3} 
                                                                     & Medium                                                                           & Very difficult                \\ \cline{2-3} 
\multirow{-3}{*}{Few (more difficult)}                               & Weak                                                                             & Moderately difficult          \\ \hline
                                                                     & Strong                                                                           & Very difficult                \\ \cline{2-3} 
                                                                     & Medium                                                                           & Moderately difficult          \\ \cline{2-3} 
\multirow{-3}{*}{Some}                                               & Weak                                                                             & Moderately to Least difficult \\ \hline
                                                                     & Strong                                                                           & Moderately difficult          \\ \cline{2-3} 
                                                                     & Medium                                                                           & Moderately difficult          \\ \cline{2-3} 
\multirow{-3}{*}{Many (less difficult)}                              & Weak                                                                             & Least difficult               \\ \hline
\end{tabular}
\captionsetup{justification=centering}
\caption{Detection Difficulty Mapping}
\end{table*}

Overall Detection Difficulty Rating: \textbf{Moderately difficult}
\end{fullwidth}