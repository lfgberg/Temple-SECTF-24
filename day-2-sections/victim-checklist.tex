\begin{fullwidth}
\section{Victim To-Do Checklist} % Top level section

If one has, or is believed to have fallen victim to an employment or tax fraud scam, it's imperative to take the proper steps to minimize resulting losses. The fraud fighters team has compiled the following checklists to assist those who have fallen victim to scams:

\subsection{Sam's Next Steps} % Second level section

\begin{enumerate}
     \item Cease contact with anyone claiming to be from Omnicell or HealthComp.
     \item Do not provide the employer with any payment information or personally identifiable information (PII) such as your address, social security number or photo.
    \item[--] A remote position should not require you to have an employee identification card. Providing a scammer with a picture as well as other forms of PII such as a social security number or address opens pathways to multiple forms of identity theft.
    \item Do not pay for any items which the company says it will reimburse you for as this matches with a common scam profile acknowledged by the \href{https://consumer.ftc.gov/articles/job-scams#what%20to%20do
}{Federal Trade Commission}
    
     \begin{quote}
	\textbf{\LARGE ``}No honest potential employer will ever send you a check to deposit and then tell you to send on part of the money, or buy gift cards with it. That’s a fake check scam. The check will bounce, and the bank will want you to repay the amount of the fake check.\textbf{''}

\hfill--- Federal Trade Commission \autocite{Hebert:2024}
 
    \end{quote}

    \item Since minimal information has been provided, the likeliness of identity fraud occurring is slim, but not null; consider investing in a identity theft protection such as Aura or Norton Security's Lifelock.
    \item Report phishing scam to \href{mailto:phishing@irs.gov}{phishing@irs.gov} and report this incident to the \href{ReportFraud.ftc.gov}{Federal Trade Commission}
    
\end{enumerate}

\subsection{Employment Scam Victim To-Do Checklist} % Second level section

\begin{enumerate}
     \item Stop all forms of communications with scammer. However do not delete any emails or texts messages sent to you as it can help serve as evidence when building a case with your local police
     \item Report the employer on the job board platform which you were solicited.
     \item If scam involved impersonation of a real company, contact their HR department to inform them of the scam.
     \item Monitor common signs of identity theft including:
    \item[--] Credit cards opened in your name.
    \item[--] Loan application denials sent your home.
    \item[--] Debt Collection notices via mail or phone.
    \item[--] Tax Return already filed by someone other than yourself.
    \item Help the government build a case and track down scammers by reporting your incident to the Federal Trade Commission (FTC).
    \item If payment was involved, contacted the involved financial institution ASAP to increase chances of conflict resolution.
    \item Close all new accounts opened in your name.
    \item Report all phishing scams to \href{mailto:phishing@irs.gov}{phishing@irs.gov}
    \item Report all monetary loss to the \href{https://www.tigta.gov/hotline?type=IRSScamsandFraud}{Treasury Inspector General Administration} and the \href{https://reportfraud.ftc.gov/#/}{Federal Trade Commission}
    \item Submit an IRS Identity Theft Affidavit (Form 14039).
    \item Initiate a credit freeze.
    \item Update online security measures.
    \item Identity theft monitoring.
    \item Seek legal advice.

    
\end{enumerate}




\end{fullwidth}