\begin{fullwidth}
\section{Gameplan} % Top level section
\subsection{Goal}
The goal for tomorrow is to inform Sam of the evidence that we collected, and how it relates to their job offer. Ultimately, we would like Sam to understand the risks of proceeding with the job offer knowing that it is likely a scam.
\subsection{Current Situation}
In the current situation, we realize that the job offer is more than likely a scam. Knowing this information, we have requested that Sam hold off on continuing sending the potential employer information.

For meeting with Sam tomorrow, we believe that we have the following strengths, weaknesses, opportunities, and threats (SWOT) leading up to our meeting:

\subsubsection{Strengths}
Going into the meeting, we find that there is more than conclusive evidence that points towards this being a job offer scam. Particularly, we find that the report of the BBB scam listing, our rating of the communications on the NIST phish scale, and our other evidence is convincing for one to understand that the job is likely a scam.

We also requested that Sam wait until tomorrow before proceeding with communication with the potential employer. We believe that this may wear down some of the excitement that she has and allow her to think about the red flags that we informed her of during the interview. If this does occur, we will be in a stronger position to inform her of the risks associated with continuing in this employment opportunity.

\subsubsection{Weaknesses}
We realize that Sam is very excited to get a high paying job with great benefits. The potential reward may blind her from understanding the facts of the situation and realizing that this is more than likely a phishing scam.

\subsubsection{Opportunities}
Since we have analysed and compiled information about many of the job scams and applied it to her experience in the hiring process with her potential employer, we believe that this puts us in a strong position to persuade her that this opportunity is a scam.

\subsubsection{Threats}
It is possible that Sam receives a message from the employer that requests that she urgently completes a task for them. Should this happen, she may panic and send personally identifiable information or money to the scammers before we are able to intervene.

\subsection{Strategies}
We recognize that Sam is excited to receive a high paying job and will be disappointed when we inform her that the job is likely a scam. She may also choose to not believe us and proceed.

Knowing this, we will proceed with the following strategies:
\subsubsection{Empathy}
We plan on approaching Sam with empathy and understanding. We have her best interests at heart and will try to prevent her from being harmed by the scammers.
\subsubsection{Evidence}
In order to convince Sam that the job is a scam, we will use all of the evidence that we compiled and present it to Sam in a persuasive way.
\subsubsection{Train}
Since Sam may be targeted by a scam in the future, we will train her on how to identify scams. Knowing this information, she should be able to better spot scams moving forward.
\subsubsection{Provide Resources}
We will provide Sam with resources to keep up to date on the current scams. We will also provide her with resources should she have had any data compromised in this incident or in any future incidents. These resources include the victim checklist that we developed. She can use this to minimize the risk of any potential identity fraud or financial loss.

\subsection{Communication Plan}
Knowing our where we currently stand and what strategies that we should employ, we will take the following actions to communicate out points to Sam:

\subsubsection{Build Trust}
In order for Sam to believe us, we must establish trust with her so that she considers our points. In order to do this, we will inform her that we are experts in the field of cybersecurity and have studied many scam cases.
\subsubsection{Understand Perspective}
We understand that Sam really want to get a job and doesn't believe that she is part of a scam. In order to bring our points across to her, we will use the evidence that we acquired and question why she thinks that it is not a scam. From there, we can then provide her with resources to find a legitimate job.
\subsubsection{Share Evidence}
Once we build trust and recognize that we understand her perspective, we will then share the evidence that we acquired for this case. This will then allow her to form her own opinion on the case.
\subsubsection{Provide Resources}
We will provide Sam with resources that inform her on what she should do now to resolve any potential breaches related to this scam and also provide her with resources to learn about scams and prevent her from being scammed in the future.
\subsubsection{Encourage Caution}
Should Sam choose to move forward with the job against our wishes, we will encourage her to exercise caution before performing any tasks for the potential employer. In particular, we will encourage her to thoroughly think over the situation before spending money or sharing personal information.

\end{fullwidth}